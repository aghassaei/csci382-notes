\documentclass{article}

% From Hao Ren's Math Notes Template (Overleaf) 
%%%%%%%%%%%%%%%%%%%%%%%%%%%%%%%%%%%%%%%%%%%%%%
\usepackage{amsmath, amsthm, amssymb, amsfonts}
\usepackage{thmtools}
\usepackage{graphicx}
\usepackage{setspace}
\usepackage{geometry}
\usepackage{float}
\usepackage{hyperref}
\usepackage[utf8]{inputenc}
\usepackage[english]{babel}
\usepackage{framed}
\usepackage[dvipsnames]{xcolor}
\usepackage{tcolorbox}
\usepackage{listings}
\usepackage{tikz-qtree}
\usepackage{xcolor}

% Layout ------------------------
\geometry{
    left=1in,
    right=1in,
    textheight=9in,
    textwidth=5.5in,
    top=1in,
    headheight=12pt,
    headsep=25pt,
    footskip=30pt
}

\newcommand{\HRule}[1]{\rule{\linewidth}{#1}}

\setstretch{1.2}



% Colors -----------------------------------
\colorlet{LightGray}{White!90!Periwinkle}
\colorlet{LightOrange}{Orange!15}
\colorlet{LightGreen}{Green!15}

% Theorems -------------------------------

\declaretheoremstyle[name=Theorem,]{thmsty}
\declaretheorem[style=thmsty,numberwithin=section]{theorem}
\tcolorboxenvironment{theorem}{colback=LightGray}

\declaretheoremstyle[name=Definition,]{thmsty}
\declaretheorem[style=thmsty,numberwithin=section]{definition}
\tcolorboxenvironment{definition}{colback=LightGray}

\declaretheoremstyle[name=Proposition,]{prosty}
\declaretheorem[style=prosty,numberlike=theorem]{proposition}
\tcolorboxenvironment{proposition}{colback=LightOrange}

\declaretheoremstyle[name=Principle,]{prcpsty}
\declaretheorem[style=prcpsty,numberlike=theorem]{principle}
\tcolorboxenvironment{principle}{colback=LightGreen}

\declaretheoremstyle[name=Corollary,]{prcpsty}
\declaretheorem[style=prcpsty,numberlike=theorem]{corollary}
\tcolorboxenvironment{corollary}{colback=LightGreen}

\declaretheoremstyle[name=Intuition,]{prcpsty}
\declaretheorem[style=prcpsty,numberlike=theorem]{intuition}
\tcolorboxenvironment{intuition}{colback=LightGreen}

% \begin{theorem}
%     This is a theorem.
% \end{theorem}

% \begin{proposition}
%     This is a proposition.
% \end{proposition}

% \begin{principle}
%     This is a principle.
% \end{principle}

%%%%%%%%%%%%%%%%%%%%%%%%%%%%%%%%%%%%%%%%%%%%%%%%%


% Code formatting --------------------

% Default color is purple
\newcommand{\code}[2][purple]{\texttt{\textcolor{#1}{#2}}}
% Example: \code[blue]{if (condition) \{ do something; \}}

\newcommand{\tand}{\text{ and }}

\lstset{
  language=Python,
  aboveskip=1pc,
  belowskip=1pc,
  basicstyle={\footnotesize\ttfamily},
  numbers=left,
  showstringspaces=false,
  numberstyle={\tiny\color{gray}\ttfamily},
  keywordstyle={\color{dkblue}\ttfamily},
  commentstyle={\color{dkgreen}\ttfamily},
  stringstyle={\color{mauve}\ttfamily},
}


% Shapes ------------------------------
\tikzset{
  % every node/.style={minimum width=2em,draw,circle},
  % level 1/.style={sibling distance=2cm},
  level distance=1cm,
  edge from parent/.style=
  {draw,edge from parent path={(\tikzparentnode) -- (\tikzchildnode)}},
}
\usetikzlibrary{shapes}




%%% Environments (Zajj D) %%%--------------------------------------------------------- 
\newenvironment{ans}{\medskip \paragraph*{\emph{Answer}.}}{\hfill \break  $~\!\!$ \dotfill \medskip }
\newenvironment{sketch}{\medskip \paragraph*{\emph{Proof sketch}.}}{ \medskip }
\newenvironment{summary}{\medskip \paragraph*{\emph{Summary}.}}{  \hfill \break  \rule{1.5cm}{0.4pt} \medskip }
\newcommand\Ans[1]{\hfill \emph{Answer:} {#1}}


%%% Alphabets (Zajj D) %%%---------------------------------------------------------
%%% Some shortcuts for my commonly used special alphabets and characters.
\def\cA{\mathcal{A}}\def\cB{\mathcal{B}}\def\cC{\mathcal{C}}\def\cD{\mathcal{D}}\def\cE{\mathcal{E}}\def\cF{\mathcal{F}}\def\cG{\mathcal{G}}\def\cH{\mathcal{H}}\def\cI{\mathcal{I}}\def\cJ{\mathcal{J}}\def\cK{\mathcal{K}}\def\cL{\mathcal{L}}\def\cM{\mathcal{M}}\def\cN{\mathcal{N}}\def\cO{\mathcal{O}}\def\cP{\mathcal{P}}\def\cQ{\mathcal{Q}}\def\cR{\mathcal{R}}\def\cS{\mathcal{S}}\def\cT{\mathcal{T}}\def\cU{\mathcal{U}}\def\cV{\mathcal{V}}\def\cW{\mathcal{W}}\def\cX{\mathcal{X}}\def\cY{\mathcal{Y}}\def\cZ{\mathcal{Z}}

\def\AA{\mathbb{A}} \def\BB{\mathbb{B}} \def\CC{\mathbb{C}} \def\DD{\mathbb{D}} \def\EE{\mathbb{E}} \def\FF{\mathbb{F}} \def\GG{\mathbb{G}} \def\HH{\mathbb{H}} \def\II{\mathbb{I}} \def\JJ{\mathbb{J}} \def\KK{\mathbb{K}} \def\LL{\mathbb{L}} \def\MM{\mathbb{M}} \def\NN{\mathbb{N}} \def\OO{\mathbb{O}} \def\PP{\mathbb{P}} \def\QQ{\mathbb{Q}} \def\RR{\mathbb{R}} \def\SS{\mathbb{S}} \def\TT{\mathbb{T}} \def\UU{\mathbb{U}} \def\VV{\mathbb{V}} \def\WW{\mathbb{W}} \def\XX{\mathbb{X}} \def\YY{\mathbb{Y}} \def\ZZ{\mathbb{Z}}  

\def\fa{\mathfrak{a}} \def\fb{\mathfrak{b}} \def\fc{\mathfrak{c}} \def\fd{\mathfrak{d}} \def\fe{\mathfrak{e}} \def\ff{\mathfrak{f}} \def\fg{\mathfrak{g}} \def\fh{\mathfrak{h}} \def\fj{\mathfrak{j}} \def\fk{\mathfrak{k}} \def\fl{\mathfrak{l}} \def\fm{\mathfrak{m}} \def\fn{\mathfrak{n}} \def\fo{\mathfrak{o}} \def\fp{\mathfrak{p}} \def\fq{\mathfrak{q}} \def\fr{\mathfrak{r}} \def\fs{\mathfrak{s}} \def\ft{\mathfrak{t}} \def\fu{\mathfrak{u}} \def\fv{\mathfrak{v}} \def\fw{\mathfrak{w}} \def\fx{\mathfrak{x}} \def\fy{\mathfrak{y}} \def\fz{\mathfrak{z}}
\def\fgl{\mathfrak{gl}}  \def\fsl{\mathfrak{sl}}  \def\fso{\mathfrak{so}}  \def\fsp{\mathfrak{sp}}  
\def\GL{\mathrm{GL}} \def\SL{\mathrm{SL}}  \def\SP{\mathrm{SL}}


\def\aa{\mathbf{a}} \def\bb{\mathbf{b}} \def\cc{\mathbf{c}} \def\dd{\mathbf{d}} \def\ee{\mathbf{e}} \def\ff{\mathbf{f}}
\def\gg{\mathbf{g}}
\def\hh{\mathbf{h}} \def\ii{\mathbf{i}} \def\jj{\mathbf{j}} \def\kk{\mathbf{k}}
\def\ll{\mathbf{l}}
\def\mm{\mathbf{m}} \def\nn{\mathbf{n}} \def\oo{\mathbf{o}} \def\pp{\mathbf{p}} \def\qq{\mathbf{q}} \def\rr{\mathbf{r}} \def\ss{\mathbf{s}} \def\tt{\mathbf{t}} \def\uu{\mathbf{u}} \def\vv{\mathbf{v}} \def\ww{\mathbf{w}} \def\xx{\mathbf{x}} \def\yy{\mathbf{y}} \def\zz{\mathbf{z}}




% \def\<{\langle} \def\>{\rangle}
\def\({\<\!\<}\def\){\>\!\>}
\def\Aut{\mathrm{Aut}}
\def\dim{\mathrm{dim}} 
\def\End{\mathrm{End}} 
\def\ev{\mathrm{ev}} 
\def\half{\hbox{$\frac12$}}
\def\img{\mathrm{img}}
\def\id{\mathrm{id}} 
\def\sgn{\mathrm{sgn}}
\def\supp{\mathrm{supp}}
\def\tr{\mathrm{tr}} 
\def\vep{\varepsilon}
\def\f{\varphi}


% Arrows:
\newcommand\xdhrightarrow[2][]{%
  \mathrel{\ooalign{$\xrightarrow[#1\mkern4mu]{#2\mkern4mu}$\cr%
  \hidewidth$\rightarrow\mkern4mu$}}
}
%\newcommand\dhrightarrow{%
%  \mathrel{\ooalign{$\rightarrow$\cr%
%  $\mkern3.5mu\rightarrow$}}
%}
\def\dhrightarrow{\twoheadrightarrow}
\def\dhleftarrow{\twoheadleftarrow}

% % Arrays:
% \newenvironment{amatrix}[1]{%
%   \left(\begin{array}{@{}*{#1}{c}|c@{}}
% }{%
%   \end{array}\right)
% }

% \newenvironment{sys}{%
%   \left\{\begin{array}{@{}r@{\ }l}
% }{%
%   \end{array}\right.
% }


\begin{document}

% Cover Page and ToC
%%%%%%%%%%%%%%%%%%%%%%%%%%%%%%%%%%%%%%%%%%%%

\title{ \normalsize \textsc{}
		\\ [2.0cm]
		\HRule{1.5pt} \\
		\LARGE \textbf{\uppercase{Algorithms \& Data Structures}
		\HRule{2.0pt} \\ [0.6cm] \LARGE{CSCI 382} \vspace*{10\baselineskip}}
		}
\date{}
\author{\textbf{Erica Blum} \\ 
		2023\\
		Notes by Aliya Ghassaei}

\maketitle
\newpage

\tableofcontents
\newpage

\section{Asymptotic notation}
    \subsection{Definitions}

        \subsubsection{Big-$O$}
            \textbf{Intuition: }If $f(n) \in O(g(n))$ then $f(n)$ grows no faster than $g(n)$. $g$ is an asymptotic upper bound for $f$.
            \begin{definition}
                $O(g(n)) :=  \{f(n) : \exists ~ c, n_0 > 0 ~|~ 0 \leq f(n) \leq c \cdot g(n) \text{ for all } n \geq  n_0 \} $
            \end{definition}

        \subsubsection{Big-$\Omega$}
            \textbf{Intuition: }If  $f(n) \in \Omega(g(n))$, $f(n)$ grows at least as fast as $g(n)$. $f$ is a lower bound on $f$.
            \begin{definition}
                $\Omega(g(n)) := \{f(n) : \exists ~ c,n_0 > 0 ~|~ 0 \leq   c \cdot g(n) \leq f(n) \text{ for all } n \geq n_0\}$
            \end{definition}

        \subsubsection{Big-$\Theta$}
            \textbf{Intuition: }$f(n)$ grows at the same rate/within a constant factor of $g(n)$
            \begin{definition}
                $\Theta(g(n)) := \{ f(n) : \exists ~ c_1, c_2, n_0 > 0 ~|~ 0 \leq c_1 \cdot g(n) \leq f(n) \leq c_2 \cdot g(n) \text{ for all } n \geq  n_0 \}$
            \end{definition}
            \begin{theorem}
                For any two functions $f(n)$ and $g(n)$, $f(n) = \Theta(g(n)) \iff f(n) = O(g(n)) $ and $f(n) = \Omega(g(n))$
            \end{theorem}


        \subsubsection{Little-$o$}
        \begin{definition}
            $o(g(n)) := \{ f(n) : \text{for any }c > 0, \exists ~ n_0 >0 ~|~ 0 \leq f(n) < c \cdot g(n) \text{ for all } n \geq  n_0 \}$
        \end{definition}
        \subsubsection{Little-$\omega$}
        \begin{definition}
            $\omega(g(n)) := \{ f(n) : \text{for any }c > 0, \exists ~ n_0 >0 ~|~ 0 \leq c \cdot g(n) < f(n) \text{ for all } n \geq  n_0 \}$
        \end{definition}
        \begin{theorem}
            $f(n) \in \omega g(n) \iff g(n) \in o(f(n))$
        \end{theorem}

        \subsection{Proofs strategies}
            \subsubsection{Proving from definition}
            This means that you'd use the definition to prove that a function $g$ is in some complexity class.\\
            \textbf{Steps: }
            \begin{enumerate}
                \item 
            \end{enumerate}
            \textbf{Example: }

            \subsubsection{Proving using limit properties}
            Use the following properties:
            \subsubsection{Helpful properties:}
            \textbf{Transitivity: }
            \begin{align*}
                f(n) &= \Theta(g(n)) \tand g(n) = \Theta(h(n)) &\implies f(n) &= \Theta(h(n))\\
                f(n) &= O(g(n)) \tand g(n) = O(h(n)) &\implies f(n) &= O(h(n))\\
                f(n) &= \Omega(g(n)) \tand g(n) = \Omega(h(n)) &\implies f(n) &= \Omega(h(n))\\
                f(n) &= o(g(n))) \tand g(n) = o(h(n)) &\implies f(n) &= o(h(n))\\
                f(n) &= \omega(g(n)) \tand g(n) = \omega(h(n)) &\implies f(n) &= \omega (h(n))
            \end{align*}
            \textbf{Reflexivity: }
            \begin{align*}
                f(n) &= \Theta(f(n))\\
                f(n) &= O(f(n))\\
                f(n) &= \Omega(f(n))
            \end{align*}
            \textbf{Symetry: }
            \begin{align*}
                f(n) = \Theta (g(n)) \iff g(n = \Theta(f(n)))
            \end{align*}
            \textbf{Transpose symetry: }
            \begin{align*}
                f(n) = O(g(n)) \iff g(n) = \Omega(f(n))\\
                f(n) = o(g(n)) \iff g(n) = \omega (f(n))
            \end{align*}

            \textbf{Limit laws: }

            {\centering
            	\begin{tabular}{ll}
            	$\lim_{n\to\infty}f(n)/g(n) \in[0,\infty)$ &	$\Rightarrow f= O(g)$ \\
            	$\lim_{n\to\infty}f(n)/g(n) \in (0,\infty] $ &		$\Rightarrow f= \Omega(g)$ \\
            	$\lim_{n\to\infty}f(n)/g(n) \in (0,\infty)$  &	$\Rightarrow f= \Theta(g)$ \\
            	$\lim_{n\to\infty}f(n)/g(n) = 0$  &	$\Rightarrow f= o(g)$ \\
            	$\lim_{n\to\infty}f(n)/g(n) = \infty$  &	$\Rightarrow f= \omega(g)$
            	\end{tabular}\par
            }


        \subsubsection{Recurrences (substitution, recurrence trees, master theorem)}
        \begin{enumerate}
            \item given recurrance relation, prove that it's in some class
            \item method: (recurrence tree) -> guess, -> substitution method
            \item look on recurrence sheet 
            \item ref sheet will have mast theorem, properties or relations for exponents and logarithms, summations
        \end{enumerate}
        \subsubsection{Models of computation (comparison model, random access model)}
    \subsection{Solving problems by reducing to searching (choose + apply a data structure)
}


\section{Interfaces \& implementations}

    \textbf{Interfaces:} set, sequence, priority queue\\
    \textbf{Implementations:} arrays, sorted arrays, linked lists, hash tables, BSTs, red-black trees, min/max heap

\subsubsection{Priority queue interface}
\begin{enumerate}
    \item heap
\end{enumerate}
\subsection{Solving problems by reducing to sorting (choose + apply a sorting algorithm)}
\begin{enumerate}
    \item Mergesort
    Insertion sort
    Selection sort
    Heapsort
    
\end{enumerate}
\subsection{Solving problems by designing a new algorithm}
\begin{enumerate}
    \item Brute force
    “Decrease and conquer”
    Divide and conquer
\end{enumerate}


\newpage
\section{Introduction}
    \subsection{Vocabulary}
        \begin{enumerate}
            \item interface: 
            \item implementation
            \item array-based (example: array/list)
            \item static: region of memory stays static
            \item pointer-based (ex: linked list)
            \item dynamic
        \end{enumerate}

\section{Data Structures}

sequence data structute operations in O\\
\begin{tabular}{|c||c|c|c|c|c|}
    \hline
    Implementation & \code{build(X)} & \code{get\_at}/\code{set\_at} & \code{insert\_first}/\code{delete\_first}& insert/delete last & insert at/delete at \\
    \hline
    array & $n$ & 1 & $n$ & $n$ & $n$ \\
    linked list & $n$ & $n$ & 1 &$n$ & $n$ \\
    dynamic array & $n$ & 1 & $n$ & $1_{(a)}$ & $n$ \\
    binary tree & $n \log n$ & $h$ & $h$ & $h$ & $h$ \\

    \hline
\end{tabular}\\
\vspace{.5cm}



Set data structute operations in O\\
\begin{tabular}{|c||c|c|c|c|c|}
    \hline
    Implementation & \code{build(X)} & \code{find(k)} & \code{insert(x)}/\code{delete(x)} & \code{find\_min/max()} & \code{find\_prev/next} \\
    \hline
    array & $n$ & $n$ & $n$ & $n$ & $n$ \\
    Sorted array & $n \log n$ & $n$ & $n$ &1 & $\log n$ \\
    direct access array & $u$ & 1 & $1$ & $u$ & $u$ \\
    Hash table & $n_{(e)}$ & $1_{(e)}$ & $1_{(a)(e)}$ & $n$ & $n$ \\
    binary tree & $n$ & $h$ & $h$ & $h$ & $h$ \\
    \hline
\end{tabular}
    \subsection{Sequences}
        Given a collection of elements called \code{X}, store elements $x_0, \dots, x_{n-1}$
        \subsubsection{Static Operations}
            \begin{enumerate}
                \item \code{build(X)} - builds the sequence from \code{X}
                \item \code{iter\_seq()} output items in order
                \item \code{len()} -  returns the number of elements in \code{X}
                \item \code{get\_at(i)} - returns item at index $i$
                \item \code{set\_at(i, x)} - insert $x$ at location $i$ 
            \end{enumerate}
        \subsubsection{Dynamic Operations}
        Dynamic sequences use static sequence operations in addition to these operations:
            \begin{enumerate}
                \item \code{insert\_at(i, x)} - inserts item $x$ at location $i$
                \item \code{delete\_at(i)} - deletes and returns item at location $i$
            \end{enumerate}
    \subsection{Sets}
        \subsubsection{Static Operations}
            \begin{enumerate}
                \item \code{build(X)} - builds set
                    \subitem - with array implementation, \code{build(X)} $\in O(n)$ where $n = |X|$
                \item \code{len()} - returns $|X|$
                \item \code{find(k)} - returns item with key $k$
                    \subitem - with array implementation, \code{find(k)} $\in O(n)$ where $n = |X|$
                \item \code{find\_min()} - returns item with smallest key
                \item \code{find\_max()} - returns item with largest key
                \item \code{find\_next(k)} - return next smallest item from $k$ (?)
                \item \code{find\_prev(k)} - return next largest item from $k$ (?)
                \item with array, min, max, prev, next all $\in O(n)$
    
            \end{enumerate}
        \subsubsection{Dynamic Operations}
            \begin{enumerate}
                \item \code{insert\_at(x, i)} - insert element $x$ at index $i$
                \item \code{delete\_at(i)} - delete and return element at $i$
                \item with array, insert/delete $\in O(n)$
            \end{enumerate}
        \subsubsection{Running times}
            \begin{enumerate}
                \item with sorted array, $\code{find(k)}, \code{prev}, \code{next} \in O(\log n)$, insert/delete in n, find min/max in 1, build is nlogn
            \end{enumerate}


    \subsection{Direct access table}
        \subsubsection{Example}
        \begin{enumerate}
            \item items are Reed students
            \item ``keyspace'' is all possible 8-digit ID numbers
            \item index by Reed ID makes $\code{find(k)} \in O(1)$, but wastes space
        \end{enumerate}

    \subsection{Hash tables}
Before: sorting and searching using only comparisons on items.\\
Now: use keys in more complex ways arrow direct access table\\

\begin{verbatim}
    h : {ids} -> {0, ... , m-1}
    # assume m = Theta(m)
\end{verbatim}
store items as determined by h(key)\\
cannot be a bijection\\
collision: things in ids that map to same idem in codomain\\
solution: instead of storing item in table, store pointer to another data structure\\
``Chaining'' to some other data structure\\
\begin{verbatim}
    def hash.find(k):
    digest = h(k)
    c = get(h.direst) -> go find k?
\end{verbatim}
\subsubsection{Running time }
ideally: all chains are constant size so then all operations on chains are constant size\\
\\
chose m,h such that all chains end up with about n/m approx O(1) things in them\\
\subsubsection{Example}
modular division: \verb|h(k) = k mod(m)|.\\
Problem: only as even as keys themselves\\
if too even, might get mapped to same thing\\
\subsubsection{Better example: universal family of hash functions}
example of one family:\\
defien $h_{ab}(k) = (((ak +b)\mod p)\mod m)$ where $p$ is a large prime and 
$a,b$ chosen randomly from 0 to p-1\\
Formally, define family $\cH(p, m) = \{h_{ab}(k) ~|~ a, b \in [0 \dots p-1] \text{ and } a \neq 0 \}$
desired property: for some $h \in \cH$, Pr that any pairs of keys collide [$h(k_1)= h(k_2)$] $\leq \frac{1}{m}$ for all k1, k2 such that k1 not equal k2 
\\
parallel concept for sorting comparison model\\
lower bound Omega (n log n)
\subsection{Binary Trees}
\section{Running Time}
\section{Correctness}
\subsection{Loop invariants}
\section{Recursion}
running time for recursive algo
"solving a recurrance"
- substitution method
- recurrence tree
- master theorem
\section{Limits}
\subsection{Limit Laws}

% L’Hospital’s Rule
% suppose:
% \[ \lim_{x\to\a} f(x)/g(x) = 0/0 \] or \[ \lim_{x\to\a} f(x)/g(x) = frac{+- infinity}{+- infinity} \]
% where 
% a
%  can be any real number, infinity or negative infinity. In these cases we have,
%  \lim_{x\to\a} f(x)/g(x) =\lim_{x\to\a} f(x)'/g(x)' =

\section{Sorting}
\subsection{Vocab}
in place or not
depends on situation, some algs better random, sorted, semi-random, etc
\subsection{Permutation Sort}
sorting

permutation sort
- input unsorted static array A
- outputs a sorted permutation of A called B
% - sorted means : B[i-1] \leq B[i] (values of keys, but when numbers dont need keys)


1. generate all permutations of A -> n!
2. for each permutation, check if it is sorted. if yes return permutation

\subsection{Selection Sort}
selection sort (A, i)
1. finds biggest item in A[:i]
2. swaps biggest thing with thing at A[i]
3. recurse on A[:i-1]

\subsection{Comparison}
\subsection{Merge Sort}
\begin{verbatim}
    if n = 0, 1: done
    otherwise:
        split array into Left, Right
        MergeSort(Left), MergeSort(Right)
        Merge(Left, Right)
\end{verbatim}
\begin{enumerate}
    \item specificity: array A, p starting point, r ending point
    \item if p geq r (if p equal to r, then only one element and return)
    \item convention: A[a:b] = [] if b less than A
    \item a:b means including both end points
    \item define midpoint $q = (p+r)/2$ (floor of the averge)
    \item Left = A[p:q]
    \item Right A[q+1:r]
    \item MergeSort(A, p, q)
    \item MergeSort(A, q+1, r)
    \item Merge(A, p, q, r)
\end{enumerate}
\begin{enumerate}

\item comparisons in constant time? \begin{enumerate}
    \item mostly comparing integers
    \item assume $i \leq j$ in constant time
    \item might be more complex
    \item    
\end{enumerate}
\end{enumerate}

% Week 5
\section{Sept 25th}
\subsection{review}
\begin{enumerate}
    \item direct access Arrays
    \item hash tables
\end{enumerate}
% Concepts so far
% \subsection{Running time (how to count)}
% \begin{enumerate}
%     \item big o
%     \item 
% \end{enumerate}
% \subsection{}
% \subsection{recurrences}
% \subsection{models of computation - which operations are O(1), word ram}

\section{Binary Search Trees}
O(h) time where h is height\\
usually use these for sets\\
example: set us cs profs\\
name/id\\
adam/560\\
charlie/703\\
erica/998\\
greg/997\\
jim/100\\
include bst from notes\\
properties: for any X, keys in x.left leq x.key leq keys x.right\\
can also store a sequence in this way\\
ex 2: grocery list\\
apples, bananas, cereal, dish soap, eggs\\
order doesn't matter\\
bst property: for any L[i], all items in L[i].left appear before L[i]\\
all items in right subtree appear after L[i]
is balanced, height is at most O(log n)\\
when balanced, called red-black trees\\
\\
\\
\subsection{non-modifying operations}
\begin{enumerate}
    \item find min - all the down left
    \item find max - all the way down right 
    \item kind k - compare at every node
    \item 
\end{enumerate}
\subsection{modifying properties}
\begin{enumerate}
    \item insert item: traverse, then insert as a leaf
    \item delete: find successor and swap 16 with 17, then remove old leaf (find a safe leaf to remove)
    \item 
\end{enumerate}
balanced if even under dynamic operations it maintains height O log N\\
red-black tree\\
\begin{enumerate}
    \item dummy nodes
    \item imagine single dummy node "tree ends"
    \item stop drawing them
\end{enumerate}
loop invariants\\
different cases for red black trees\\
\section{priority queues and heaps}
\subsection{Priority Queue Interface}
new: delete max - remove item with highest priority\\
find max = find highest priority item\\
supports subset of set interface operations\\
optemized for finding min or max, we will focus on max priority\\
also has insert and build
\subsection{priority queue sort}
algo in slides\\
simple because you make the ds do it\\
\subsection{binary heaps}
\subsubsection{infer tree from aray}
require: completely full in upper levels (until last level)\\
binary trees with n things -> arrays of length n\\
bottom level should be left justified\\
diferent than traversal ordering function\\
root is index 0\\
finding the left child of i is just at index 2i+1\\
right(i) = 2i+2\\
parent(i) = floor((i-1)/2)\\
max heap properties:
\begin{enumerate}
    \item at node l, the value of thing at position l is going to be at least the value its children\\
    \item 
\end{enumerate}
tree does not live in memory, just array\\
we want higher importance nodes to be closer to the top\\
bottom layer is less than all elements above it\\
example 1 from slides: not a max heap: 15 is a node that is is not greater than or equal to both of its children\\
\\
inserting(x)\\
-append x to array: whole function is in O(logn)\\
-fix problems in heap\\
fixing: swap up with parents until max heap rule is followed:\\
check that val of parent geq val node, if no, swap and keep going\\
delete-max():\\
this deletes the root! \\
1. swap root with item at node number n-1, then delete thing at n-1\\
the fix downwards:\\
- start at root\\
- swap it with the greater key in its children, then continue\\
this is o of log n time
\section{Cheat Sheets}


% % \begin{table}[h]
% %     \small
% %     \begin{tabular}{l ||| ccccc|||}
    
% %     \multirow{3}{*}{\begin{tabular}[c]{@{}l@{}}Sequence\\ Data Structure\end{tabular}} &
% %     \multicolumn{5}{c|||}{Operations $O(\cdot)$} \\ \cline{2-6}
% %       & \multicolumn{1}{c||}{Container}  
% %       & \multicolumn{1}{c||}{Static} 
% %       & \multicolumn{3}{c|||}{Dynamic}  \\ \cline{2-6} 
% %       & \multicolumn{1}{c||}{\footnotesize${\mathtt{build(X)}}$} 
% %       & \multicolumn{1}{c||}{\footnotesize\begin{tabular}[c] {@{}c@{}}${\mathtt{get\_at(i)}}$\\ ${\mathtt{set\_at(i,x)}}$\end{tabular}}
% %       & \multicolumn{1}{c|}{\footnotesize\begin{tabular}[c]{@{}c@{}}${\mathtt{insert\_first(x)}}$\\ ${\mathtt{delete\_first()}}$\end{tabular}}
% %       & \multicolumn{1}{c|}{\footnotesize\begin{tabular}[c]{@{}c@{}}${\mathtt{insert\_last(x)}}$\\ ${\mathtt{delete\_last()}}$\end{tabular}} &
% %       \footnotesize\begin{tabular}[c|]{@{}c@{}}${\mathtt{insert\_at(i,x)}}$\\ ${\mathtt{delete\_at(i)}}$\end{tabular} \\ \hline
      
%     % Array   & \multicolumn{1}{c||}{\nTable{}} & \multicolumn{1}{c||}{\constTable{}} & \multicolumn{1}{c|}{\nTable{}}   & \multicolumn{1}{c|}{\nTable{}}     &  \multicolumn{1}{c|||}{\nTable{}}       \\ \hline
%     % Linked List	 & \multicolumn{1}{c||}{\nTable{}}	 & \multicolumn{1}{c||}{\nTable{}} 	& \multicolumn{1}{c|}{\constTable{}} & \multicolumn{1}{c|}{\nTable{}}   & \multicolumn{1}{c|||}{\nTable{}}    \\ \hline
%     % Dynamic Array  & \multicolumn{1}{c||}{\nTable{}}  & \multicolumn{1}{c||}{\constTable{}}  & \multicolumn{1}{c|}{\nTable{}}  & \multicolumn{1}{c|}{\customCell{\bestcolor}{$1_{(a)}$}}    &  \multicolumn{1}{c|||}{\nTable{}}    \\ \hline
%     % Binary Tree  & \multicolumn{1}{c||}{\nlognTable{}}  & \multicolumn{1}{c||}{\hTable}  & \multicolumn{1}{c|}{\hTable{}}  & \multicolumn{1}{c|}{\hTable{}}    &  \multicolumn{1}{c|||}{\hTable{}}    \\ \hline
%     % \end{tabular}
%     % \end{table}
    
    
    % \begin{table}[h]
    % \normalsize
    % \begin{tabular}{l ||| ccccc|||}
    
    % \multirow{3}{*}{\begin{tabular}[c]{@{}l@{}}Set\\ Data Structure\end{tabular}} &
    % \multicolumn{5}{c|||}{Operations $O(\cdot)$} \\ \cline{2-6}
    %   & \multicolumn{1}{c||}{Container}  
    %   & \multicolumn{1}{c||}{Static} 
    %   & \multicolumn{1}{c||}{Dynamic}  
    %   & \multicolumn{2}{c|||}{Order}  \\ \cline{2-6} 
    %   & \multicolumn{1}{c||}{\footnotesize${\mathtt{build(X)}}$} 
    %   & \multicolumn{1}{c||}{\footnotesize${\mathtt{find(k)}}$}
    %   & \multicolumn{1}{c||}{\footnotesize\begin{tabular}[c]{@{}c@{}}${\mathtt{insert(x)}}$\\ ${\mathtt{delete(k)}}$\end{tabular}}
    %   & \multicolumn{1}{c|}{\footnotesize\begin{tabular}[c]{@{}c@{}}${\mathtt{find\_min()}}$\\ ${\mathtt{find\_max()}}$\end{tabular}} &
    %   \footnotesize\begin{tabular}[c|]{@{}c@{}}${\mathtt{find\_prev(k)}}$\\ ${\mathtt{find\_next(k)}}$\end{tabular} \\ \hline 
      
    % Array   & \multicolumn{1}{c||}{\nTable{}} & \multicolumn{1}{c||}{\nTable{}} & \multicolumn{1}{c||}{\nTable{}}   & \multicolumn{1}{c|}{\nTable{}}     &  \multicolumn{1}{c|||}{\nTable{}}       \\ \hline
    % Sorted Array	 & \multicolumn{1}{c||}{\nlognTable{}}	 & \multicolumn{1}{c||}{\lognTable{}} 	& \multicolumn{1}{c||}{\nTable{}} & \multicolumn{1}{c|}{\constTable{}}   & \multicolumn{1}{c|||}{\lognTable{}}    \\ \hline
    % Direct Access Array  & \multicolumn{1}{c||}{\uTable{}}  & \multicolumn{1}{c||}{\constTable{}}  & \multicolumn{1}{c||}{\constTable{}}  & \multicolumn{1}{c|}{\uTable{}}    &  \multicolumn{1}{c|||}{\uTable{}}    \\ \hline
    % Hash Table      & \multicolumn{1}{c||}{\customCell{\badcolor}{$n_{(e)}$}}  & \multicolumn{1}{c||}{\customCell{\bestcolor}{$1_{(e)}$}}                         & \multicolumn{1}{c||}{\customCell{\bestcolor}{$1_{(a)(e)}$}}  & \multicolumn{1}{c|}{\nTable{}}   &   \multicolumn{1}{c|||}{\nTable{}}                                                                                    \\ \hline
    % Binary Tree      & \multicolumn{1}{c||}{\nTable{}}  & \multicolumn{1}{c||}{\hTable{}}  & \multicolumn{1}{c||}{\hTable{}}  & \multicolumn{1}{c|}{\hTable{}}   &   \multicolumn{1}{c|||}{\hTable{}}                                                                                    \\ \hline
    % \end{tabular}
    % \end{table}
    




\subsection{Running Times for DS}
\begin{center}
    

\begin{tabular}{ |p{3cm}||p{3cm}|p{3cm}|p{3cm}|p{3cm}|  }
    \hline
    \multicolumn{5}{|c|}{Operations $O(\cdot)$} \\
    \hline
    Set Data Structure& Container& Static& Dynamic& Order\\
    \hline
    array&   AX  & ALA   &248\\
    sorted array&AL & ALB&  008\\

    \hline
   \end{tabular}
\end{center}
\subsection{orders}


\section{servers}
\begin{enumerate}
    \item servers and clients
    \item each server: some capacity
    \item each client: some workload
    \item assign so that no server goes over its capacity
\end{enumerate}
implementation
\begin{enumerate}
    \item initialize(X) whre X is set of servers
    \item connect(client, workload)
    \item disconnect(client)
    \item clients-of(server): return list of clients connected to that server
\end{enumerate}
how to do:\\
idea\\
use priority queue for clients and prioritize biggest \\
priority queue for servers so that you can find best fit the best\\
\\
notes on actual:\\
server has set of clients where key is client id and value is workload\\
priority queue storing the servers\\
store based on priorty, which is available capacity\\
\\
structures:\\
pq P:(server.id, available capacity)\\
set S: (maps server id to set of clients and pointer to si in P)
set C: maps client id to connected server and workload (if not stored as client object)

\section{Graph algorithms}
    \subsection{Definition}
        \begin{enumerate}
            \item $G = (V, E)$ where for all $(u,v) \in E, $ $u \neq v$ and no duplicate edges (for this class)
            \item $Adj+(u) := \{v \in V ~|~ (u, v) \in E\}$ (outgoing neighbors of node $u$)
            \item incoming neighbord of u Adj - (u) = set(v in V st (v, u) in E)
            \item out degree(u) = size of adj+
            \item in degree(u) = size adj -\\
        \end{enumerate}
    \subsection{Representing graphs}
    matrix or adj list\\
    storing:\\
    top level set adj of vertices\\
    lower level: adj lists Adj(u) either set or sequence (default to outgoing\\
    can be direct access array\\
    )
    \subsection{Running time}
        \begin{enumerate}
            \item linear means O(V + E) (size of those sets)
            \subitem doing something with each edge and each vertex
            \item $|E| = O(V^2)$ for simple graphs
            \subitem most edges you can have is fully connected graph
            \item undirected: |E| leq size of V choose 2 (proved on homework)
            \item directed : |E| leq 2 times (V size chose 2)
        \end{enumerate}



\subsection{Breadth-first search}


\subsection{Path problems}

\section{Log rules}
\begin{tabular}{|c|c|}
    \hline
    Product rule & $\log_b(MN)=\log_b(M)+\log_b(N)$ \\
    \hline
    Quotient rule & $\log_b\left(\dfrac{M}{N}\right)=\log_b(M)-\log_b(N)$ \\
    \hline Power rule & $\log_b(M^p)=p\cdot\log_b(M)$\\
    \hline Base switch rule & $\log_b(c)=\frac{1}{\log_c(b)}$\\
    \hline Base change rule & $\log_b(x)=\frac{\log_c(x)}{\log_c(b)}$\\
    \hline
    \end{tabular}




% \section{Feedback}
% \subsection{Problem set 1}
% \begin{enumerate}
%     \item Missing justification of $n \leq (log n)^(log n)$
%     \item Give a specific expression for the new $n_0, c_1, c_2$ in 
%     terms of the corresponding constants for f and g. 
%     Also, note that the definition of f(n) in O(g) (or Theta(g), 
%     or Omega(g)) only tells us that there exist constants such that 
%     the inequality holds, not necessarily equality. So in the case of O(g), 
%     we know there exist c, $n_0$ s.t. $0<=f(n)<=c g(n)$ for all $n>=n_0$, 
%     but it is not necessarily true that there exists c, $n_0$ s.t. $f(n)=c g(n)$ 
%     for all $n>=n_0 (<= vs =)$.
% \end{enumerate}
% \subsection{Problem set 2}
% \subsection{Problem set 3c}






% % \subsection{Citation}

% % This is a citation\cite{Eg}.

% % \newpage

% % % ------------------------------------------------------------------------------
% % % Reference and Cited Works
% % % ------------------------------------------------------------------------------

% % \bibliographystyle{IEEEtran}
% % \bibliography{References.bib}

% % % ------------------------------------------------------------------------------





\end{document}
